\documentclass[fleqn]{article}
\author{tbd}

\usepackage{amsmath}

\begin{document}

\title{Variably Saturated Flow in MODFLOW 6}
\maketitle

\tableofcontents

\section{Theoretical Development}
In this section ...

\subsection{Conservation of Mass}
Given a volume of porous medium $V$ bounded by a surface $\sigma$ as
shown in figure~[tbd], conservation of mass for liquid water
requires that the following equation be satisfied:
\begin{equation}
  \int_V \frac{\partial (\rho s \phi)}{\partial t} \,dV +
  \oint_\sigma \rho \mathbf{q} \cdot \mathbf{n} d\sigma -
  \int_V \rho \Gamma \,dV = 0,
  \label{eq-conservation}
\end{equation}
where
\begin{eqnarray*}
  \rho &=& \textrm{liquid density, [ML\textsuperscript{-3}]} \\
  s &=& \textrm{liquid saturation, [L\textsuperscript{0}]} \\
  \phi &=& \textrm{porosity, [L\textsuperscript{0}]} \\
  t &=& \textrm{time, [T]} \\
  \mathbf{q} &=& \textrm{vector of liquid flux per unit area, [LT\textsuperscript{-1}]} \\
  \mathbf{n} &=& \textrm{normal vector to the surface $\sigma$, [L\textsuperscript{0}]} \\
  \Gamma &=& \textrm{volumetric source-sink term accounting for liquid added to} \\
         && \textrm{($\Gamma > 0$) or taken away ($-\Gamma < 0$) from the volume $\Omega$, per unit} \\
         && \textrm{volume per unit time, [T\textsuperscript{-1}]}
\end{eqnarray*}
Equation~\ref{eq-conservation} states that the rate of change of
mass stored in $V$ must be balanced by the sum of liquid flux across
the surface boundary of $V$ and of liquid added by sources or removed
at sinks.

It is assumed that the volume $V$ is small enough that within $V$, the
liquid density ($\rho$), saturation ($s$), and porosity ($\phi$) can be
considered constant "representative" values, so that the first term of
equation~\ref{eq-conservation} can be expressed as
\begin{equation}
  \int_V \frac{\partial (\rho s \phi)}{\partial t} \,dV = 
  V \frac{\partial (\rho s \phi)}{\partial t},
\end{equation}
and the third term as
\begin{equation}
  \int_V \rho \Gamma \,dV = \rho \Gamma V.
\end{equation}
Equation~\ref{eq-conservation} then becomes
\begin{equation}
  V \frac{\partial (\rho s \phi)}{\partial t} +
  \oint_\sigma \rho \mathbf{q} \cdot \mathbf{n} d\sigma -
  \rho \Gamma V = 0.
\end{equation}

\subsection{Fluid-Flux Equation}
The fluid flux $\mathbf{q}$ is given by Darcy's law for unsaturated flow:
\begin{equation}
  \mathbf{q} = - \frac{\mathbf{k}(s)}{\mu} \cdot ( \nabla p + \rho g \nabla z ),
\end{equation}
where the permeability of the medium $\mathbf{k}$ depends on its saturation $s$,
$\mu$ is the dynamic viscosity, $p$ the pressure of the fluid, and $g$ the
gravitational acceleration. For constant density and ignoring the temperature
dependence of viscosity, this can be writtten in terms of piezometric head as
\begin{equation}
  \mathbf{q} = - \mathbf{K}(s) \cdot \nabla h,
\end{equation}
with $\mathbf{K}$ a rank-2 tensor of hydraulic conductivity. The concept of relative
permeability can be used to rewrite this expression such that its relation
to the fully saturated case becomes apparent:
\begin{equation}
  \mathbf{q} = - \mathbf{k}_\text{r}(s) \cdot
  \mathbf{K}_{\textrm{sat}} \cdot \nabla h,
\end{equation}
where the relative permeability $\mathbf{k}_r$ is a rank-2 tensor as well and
$\mathbf{K}_\text{sat}$ is the hydraulic conductivity tensor at full saturation.

For now we assume that relative permeability is isotropic (!!) and that this 
can be further simplified as
\begin{equation}
  \mathbf{q} = - k_\text{r}(s) \mathbf{K}_{\textrm{sat}} \cdot \nabla h,
\end{equation}
with $k_r$ a scalar function of saturation.

Substituting the expression for the fluid flux into equation~\ref{eq-conservation}
leads to
\begin{equation}
  V \frac{\partial (\rho s \phi)}{\partial t} -
  \oint_\sigma \rho k_\text{r}(s) \mathbf{K}_{\textrm{sat}}
  \cdot \nabla h \cdot \mathbf{n} d\sigma -
  \rho \Gamma V = 0.
\end{equation}

If the terms in the surface integral can be considered constant over each of the 
$m$ faces of an arbitrary prismatic volume element (isotropic!?), 
this can be written as:
\begin{equation}
  V \frac{\partial (\rho s \phi)}{\partial t} -
  \rho \sum_{\alpha=1}^{m} \overline{k_r K}
  \left[\frac{\partial h}{\partial n}\right]_\alpha \sigma_\alpha -
  \rho \Gamma V = 0,
  \label{eq-diff-csv}
\end{equation}
where $\sigma_\alpha$ equals the area of the $\alpha$ face and the bar indicates 
that an effective value is used for hydraulic conductivity, to be discussed 
in more detail in section~\ref{sec-cond-avg}. Additionally, the head 
gradient is assumed to be normal to the faces bounding the volume which
means that for any face $\alpha$ it follows that:
$(\nabla h)_\alpha = (\partial h/\partial n)_\alpha \mathbf{n_\alpha}$.

\subsection{Storage Term}
The change in mass of water that is held in storage is expressed by the first 
term in equation~\ref{eq-diff-csv}. Applying the chain rule results in
\begin{equation}
  \frac{\partial (\rho s \phi)}{\partial t} =
  \underbrace{s \phi \frac{\partial \rho}{\partial t}}_\text{A} +
  \underbrace{\rho \phi \frac{\partial s}{\partial t}}_\text{B} +
  \underbrace{\rho s \frac{\partial \phi}{\partial t}}_\text{C},
\end{equation}
with three distinct effects that contribute to the added mass of water: fluid
compressibility (A), change in saturation (B), and the compressibility of the
solid matrix (C). \textit{In what follows, we will assume that the porous medium
is non-deformable: $\partial \phi/\partial t = 0$. ..... not true}

In the model, the principal dependent variable used for solving the system
of equations is hydraulic head and, again using the chain rule, the equation
for change in storage can be rewritten as follows:
\begin{equation}
  \frac{\partial (\rho s \phi)}{\partial t} =
  \left(
    \rho s S_s + \rho C_m
  \right)
  \frac{\partial h}{\partial t},
  \label{eq-sto}
\end{equation}
where specific storage is given by 
\begin{equation}
  S_s = \rho g \phi \beta_c + ...,
\end{equation}
with $\beta_c$, the fluid compressibility coefficient, equal to
\begin{equation}
  \beta_c = \frac{1}{\rho}\frac{\partial \rho}{\partial p},
\end{equation}
and where we have used the assumption that the density of water depends
on pressure only: $\rho = \rho(p)$. The change in saturation resulting 
from a change in pressure head ($h_p$) is captured by the specific 
moisture capacity, which is the slope of the moisture retention curve: 
$C_m = \partial \theta / \partial h_p$.

\subsection{Variably Saturated Flow Equation}
Putting the results from previous sections together, substituting the storage terms
from equation~\ref{eq-sto} into the differential form of the balance law
given by equation~\ref{eq-diff-csv}, leads to the following nonlinear
partial differential equation for unsaturated zone flow:
\begin{equation}  
  V
  \left(
    \rho s S_s + \rho C_m
  \right)
  \frac{\partial h}{\partial t} -
  \rho \sum_{\alpha=1}^{m} \overline{k_r K}
  \left[\frac{\partial h}{\partial n}\right]_\alpha \sigma_\alpha -
  \rho \Gamma V = 0.
\end{equation}
with
\begin{eqnarray*}
  V &=& \textrm{the representative elementary volume} \\
  \rho &=& \textrm{water density} \\
  s &=& \textrm{water saturation} \\
  S_s &=& \textrm{specific storage} \\
  C_m &=& \textrm{specific moisture capacity} \\
  h &=& \textrm{hydraulic head} \\
  k_r &=& \textrm{relative permeability} \\
  K &=& \textrm{hydraulic conductivity} \\
  \sigma_\alpha &=& \textrm{surface area of face $\alpha$} \\
  \Gamma &=& \textrm{volumetric source-sink term}
\end{eqnarray*}

\subsection{Initial and Boundary Conditions}

\subsection{Analytical Functions for Soil Properties}
\subsubsection*{Water Retention Curve}
\subsubsection*{Specific Moisture Capacity}
\subsubsection*{Relative Hydraulic Conductivity}
\subsection{Hysteresis}

\section{Numerical Solution}
\subsection{Spatial Discretization}
\subsection{Intercell Averaging of Conductance Terms}\label{sec-cond-avg}
\subsubsection*{Saturated Hydraulic Conductivity}
\subsubsection*{Relative Hydraulic Conductivity}
\subsection{Temporal Discretization}
\subsection{Linearization}


\section{Model Verification}
\subsection{Analytical Results}
\subsection{Example Problems}

\end{document}