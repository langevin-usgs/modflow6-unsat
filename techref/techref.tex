\documentclass[fleqn]{article}
\author{Authors from Deltares and USGS}

\usepackage{amsmath}

\begin{document}

\title{Variably Saturated Flow in MODFLOW 6}
\maketitle

\tableofcontents

\section{Theoretical Development}
In this section ...

\subsection{Conservation of Mass}
In case of variably saturated flow, the void space is occupied by 
two fluid phases, air and water, and a full description includes
the simultaneous flow of both phases. For the process described here
it is assumed that there is no mass transfer between the fases and
the air is stagnant such that its pressure equals zero everywhere.
Under those conditions, the differential macroscopic mass balance 
equation for the water phase has the form (c.f. equation~(6.3.3)
in Bear and Cheng, 2010):

\begin{equation}
  \frac{\partial (\rho s \phi)}{\partial t} +
  \nabla \cdot (\rho \mathbf{q}) -
  \rho \Gamma = 0,
  \label{eq-balance-equation}
\end{equation}
where
\begin{eqnarray*}
  \rho &=& \textrm{liquid density, [ML\textsuperscript{-3}]} \\
  s &=& \textrm{liquid saturation, [-]} \\
  \phi &=& \textrm{porosity, [-]} \\
  t &=& \textrm{time, [T]} \\
  \mathbf{q} &=& \textrm{vector of liquid flux per unit area, [LT\textsuperscript{-1}]} \\
  \Gamma &=& \textrm{volumetric source-sink term accounting for liquid added to} \\
  && \textrm{($\Gamma > 0$) or taken away ($-\Gamma < 0$) from the volume $\Omega$, per unit} \\
  && \textrm{volume per unit time, [T\textsuperscript{-1}]}
\end{eqnarray*}
Integrating equation~\ref{eq-balance-equation} over a volume of porous
medium $V$ bounded by a surface $\sigma$, and applying Gauss's theorem
leads to:
\begin{equation}
  \int_V \frac{\partial (\rho s \phi)}{\partial t} \,dV +
  \oint_\sigma \rho \mathbf{q} \cdot \mathbf{n} d\sigma -
  \int_V \rho \Gamma \,dV = 0,
  \label{eq-conservation}
\end{equation}
where $\mathbf{n}$ equals the normal vector to the surface $\sigma$. 
Equation~\ref{eq-conservation} states that the rate of change of
mass stored in $V$ must be balanced by the sum of liquid flux across
the surface boundary of $V$ and of liquid added by sources or removed
at sinks.

The volume $V$ is assumed small enough that within it, the
liquid density ($\rho$), saturation ($s$), and porosity ($\phi$) can be
considered constant "representative" values, such that the first term of
equation~\ref{eq-conservation} can be expressed as
\begin{equation}
  \int_V \frac{\partial (\rho s \phi)}{\partial t} \,dV = 
  V \frac{\partial (\rho s \phi)}{\partial t},
\end{equation}
and the third term as
\begin{equation}
  \int_V \rho \Gamma \,dV = \rho \Gamma V.
\end{equation}
Equation~\ref{eq-conservation} now becomes
\begin{equation}
  V \frac{\partial (\rho s \phi)}{\partial t} +
  \oint_\sigma \rho \mathbf{q} \cdot \mathbf{n} d\sigma -
  \rho \Gamma V = 0.
\end{equation}

\subsection{Fluid-Flux Equation}
The fluid flux $\mathbf{q}$ is given by Darcy's law for unsaturated flow:
\begin{equation}
  \mathbf{q} = - \frac{\mathbf{k}(s)}{\mu} \cdot ( \nabla p + \rho g \nabla z ),
\end{equation}
where the permeability of the medium $\mathbf{k}$ depends on its saturation $s$,
$\mu$ is the dynamic viscosity, $p$ the pressure of the fluid, and $g$ the
gravitational acceleration. For constant density and ignoring the temperature
dependence of viscosity, this can be writtten in terms of piezometric head as
\begin{equation}
  \mathbf{q} = - \mathbf{K}(s) \cdot \nabla h,
\end{equation}
with $\mathbf{K}$ a rank-2 tensor of hydraulic conductivity. The concept of relative
permeability can be used to rewrite this expression such that its relation
to the fully saturated case becomes apparent:
\begin{equation}
  \mathbf{q} = - \mathbf{k}_\text{r}(s) \cdot
  \mathbf{K}_{\textrm{sat}} \cdot \nabla h,
\end{equation}
where the relative permeability $\mathbf{k}_r$ is a rank-2 tensor as well and
$\mathbf{K}_\text{sat}$ is the hydraulic conductivity tensor at full saturation.

For now we assume that relative permeability is isotropic (!!) and that this 
can be further simplified as
\begin{equation}
  \mathbf{q} = - k_\text{r}(s) \mathbf{K}_{\textrm{sat}} \cdot \nabla h,
\end{equation}
with $k_r$ a scalar function of saturation.

Substituting the expression for the fluid flux into equation~\ref{eq-conservation}
leads to
\begin{equation}
  V \frac{\partial (\rho s \phi)}{\partial t} -
  \oint_\sigma \rho k_\text{r}(s) \mathbf{K}_{\textrm{sat}}
  \cdot \nabla h \cdot \mathbf{n} d\sigma -
  \rho \Gamma V = 0.
\end{equation}

If the terms in the surface integral can be considered constant over each of the 
$m$ faces of an arbitrary prismatic volume element (isotropic!?), 
this can be written as:
\begin{equation}
  V \frac{\partial (\rho s \phi)}{\partial t} -
  \rho \sum_{\alpha=1}^{m} \overline{k_r K}
  \left[\frac{\partial h}{\partial n}\right]_\alpha \sigma_\alpha -
  \rho \Gamma V = 0,
  \label{eq-diff-csv}
\end{equation}
where $\sigma_\alpha$ equals the area of the $\alpha$ face and the bar indicates 
that an effective value is used for hydraulic conductivity, to be discussed 
in more detail in section~\ref{sec-cond-avg}. Additionally, the head 
gradient is assumed to be normal to the faces bounding the volume which
means that for any face $\alpha$ it follows that:
$(\nabla h)_\alpha = (\partial h/\partial n)_\alpha \mathbf{n_\alpha}$.

\subsection{Storage Term}
The change in mass of water that is held in storage is expressed by the first 
term in equation~\ref{eq-diff-csv}. To express this in terms of hydraulic
head, the principal dependent variable for solving the system of equations,
the chain rule is applied:
\begin{equation}
  \frac{\partial (\rho s \phi)}{\partial t} =
  \left[
    s \phi \frac{\partial \rho}{\partial h} +
    \rho \phi \frac{\partial s}{\partial h} +
    \rho s \frac{\partial \phi}{\partial h}
  \right]
  \frac{\partial h}{\partial t},
  \label{eq-sto-terms}
\end{equation}
with between brackets three distinct terms contributing to the water 
added to or released from storage due to changes of the hydraulic 
head: the fluid compressibility term accounting for changes in the 
fluid density $\rho$, the change in soil saturation $s$, and the (elastic) compression of the solid matrix given by the change in the porosity 
of the medium $\phi$. For any particular type of soil or geologic formation,
these variables are taken to be a function of pressure only. The 
following definitions are then introduced to restructure the right-hand 
side of equation~\ref{eq-sto-terms}:
\begin{eqnarray*}
  &&\text{Fluid compressibility coefficient: }
    \beta_c = \frac{1}{\rho} \frac{\partial \rho}{\partial p} \\
  &&\text{Specific moisture capacity: }
    C_m = \phi \frac{\partial s}{\partial \psi} \\
  &&\text{Solid matrix compressibility coefficient: }
    \alpha_c = \frac{\partial \phi}{\partial p},
\end{eqnarray*}
with $\psi$ the pressure head of the soil water: $\psi = h - z$. Now
with the approximation that the density of water depends on pressure 
only, $\rho = \rho(p)$, the change in storage can be written as
\begin{equation}
  \frac{\partial (\rho s \phi)}{\partial t} =
  \left(
    \rho s S_s + \rho C_m
  \right)
  \frac{\partial h}{\partial t},
  \label{eq-sto}
\end{equation}
where specific storage is defined as
\begin{equation}
  S_s = \rho g \left( \phi \beta_c + \alpha_c \right).
\end{equation}

\subsection{Variably Saturated Flow Equation}
Putting the results from the previous sections together, thereby
substituting the storage terms (equation~\ref{eq-sto}) into the 
differential form of the balance law (equation~\ref{eq-diff-csv}), 
leads to the following nonlinear partial differential equation for 
variably saturated flow:
\begin{equation}  
  V
  \left(
    s S_s + C_m
  \right)
  \frac{\partial h}{\partial t} -
  \sum_{\alpha=1}^{m} \overline{k_r K}
  \left[\frac{\partial h}{\partial n}\right]_\alpha \sigma_\alpha -
  \Gamma V = 0.
  \label{eq-unsat-flow}
\end{equation}
with (repeated here for convenience):
\begin{eqnarray*}
  V &=& \text{the representative elementary volume} \\
  s &=& \text{water saturation} \\
  S_s &=& \text{specific storage} \\
  C_m &=& \text{specific moisture capacity} \\
  h &=& \text{hydraulic head} \\
  \overline{k_r} &=& \text{effective relative permeability} \\
  \overline{K} &=& \text{effective hydraulic conductivity} \\
  (\partial h/\partial n)_\alpha &=& \text{head gradient normal to face $\alpha$} \\
  \sigma_\alpha &=& \text{surface area of face $\alpha$} \\
  \Gamma &=& \text{volumetric source-sink term}
\end{eqnarray*}

Note that as the medium approaches full saturation ($s\uparrow1$), 
the moisture capacity $C_m$ will go to zero and $k_r$ becomes
equal to 1. In that case, the above expression converges smoothly
to the saturated groundwater flow equation used in MODFLOW~6.
It is this concept of continuity that is used to integrate the 
variably saturated flow mechanism into the existing flow formulation.


\subsection{Initial and Boundary Conditions}\label{sec-boundary-conditions}


\subsection{Analytical Functions for Soil Properties}
\subsubsection*{Water Retention Curve}
\subsubsection*{Specific Moisture Capacity}
\subsubsection*{Relative Hydraulic Conductivity}
\subsubsection*{Hysteresis}

\section{Numerical Solution}
Equation~\ref{eq-unsat-flow}, together with the boundary conditions as
described in section~\ref{sec-boundary-conditions}, is a highly non-linear
partial differential equation that has no general closed-form solution.
Although it can be considered a more advanced description of unconfined
and partially saturated groundwater flow than the standard formulation
available in MODFLOW 6, the solution strategy is the same. The variably
saturated flow equation is discretized using a controle-volume
finite-difference (CVFD) method. The method replaces the continuum of
coordinates in equation~\ref{eq-unsat-flow} with a discrete set of 
points in space and time. The control volume is called a cell and the
point at the center of the cell a node. The collection of cells with
their connections is the model grid. The partial derivatives in the
equation can be approximated by differencing head values between 
neighbouring cells and between subsequent time values. The resulting
finite-difference equations will have terms that are a function
of the discretized dependent variable, such as $k_r = k_r(h)$ and 
$s = s(h)$. These non-linearities can be overcome using a standard
Picard iteration method\footnote{Are we planning to support non-NR
treatment of Richards flow??} or by applying the Newton-Raphson 
linearization that is known to provide more stability when solving for
cells that are under a water-table condition. For each cell in the
the grid that is under variably saturated flow conditions, the
linearized balance terms are added to the system representing the
motion equations in the total (saturated and unsaturated) simulation 
domain. This system is solved by the MODFLOW iterative solver as part
of its outer (non-linear) iteration loop.

\subsubsection*{Water Balance on a Cell}
The variably saturated flow equation in CVFD form can be used to
express the concept of water balance on a cell $n$ as
\begin{equation}
  \sum_{m \in \eta_n} Q_{n,m} + Q_{n, \Gamma} - Q_{\textrm{STO}} = 0,
  \label{eq-cell-balance}
\end{equation}
where
\begin{eqnarray*}
  Q_{n,m} &=& \text{flow rate from cell $m$ into $n$} \\
  \eta_n &=& \text{collection of cells connected to $n$} \\
  Q_{n,\Gamma} &=& \text{summed contribution from sources and sinks in the cell} \\
  Q_{\textrm{STO}} &=& \text{change in volume of water stored in $n$ per unit time}
\end{eqnarray*}
and the terms in equation~\ref{eq-cell-balance} are all of dimension
[L\textsuperscript{3}T\textsuperscript{-1}]. For every cell $n$, after
reordering, this water balance equation corresponds to the 
$n$\textsuperscript{th} row in the matrix equation
\begin{equation}
  \mathbf{A}\mathbf{h} = \mathbf{b},
  \label{eq-nonlinear-system}
\end{equation}
where $\mathbf{A}$ equals a $n \times n$ coefficient matrix, $\mathbf{h}$
the dependent variable vector at the end of the time step, and
$\mathbf{b}$ the constant terms grouped in a vector that is commonly
referred to as the right-hand side. 

For general variably saturated flow problems, the coefficient matrix is 
nonlinear and depends on the head values that are to be calculated. 
Using the Picard iteration method, equation~\ref{eq-nonlinear-system} 
can be written as
\begin{equation}
  \mathbf{\bar{A}}^{(m-1)} \mathbf{h}^{(m)} = \mathbf{b},
\end{equation}
such that $\mathbf{h}^{(m)}$ is the head vector at the new iteration level
$m$ and $\mathbf{\bar{A}}^{(m-1)}$ is the linearized coefficient matrix
that no longer depends on $\mathbf{h}^{(m)}$ but on the values from the previous
iteration level $m-1$ instead. Repeatedly formulating 
and solving the system until $\mathbf{h}^{(m)} \approx \mathbf{h}^{(m-1)}$ 
gives a numerical solution of equation~\ref{eq-nonlinear-system}.

In the following sections, discretized expressions are presented for
all the terms in equation~\ref{eq-cell-balance}. These expressions,
together with the linearization procedure and a discrete
formulation of available initial conditions and boundary conditions, 
are required to numerically solve the process of variably saturated 
flow within the MODFLOW 6 framework.

\subsubsection*{Newton-Raphson Linearization}
An alternative method to ...
\begin{equation}
  \mathbf{J}^{k-1} \Delta \mathbf{h}^k = -\mathbf{r}^{k-1}
\end{equation}

\begin{equation}
  \mathbf{J}^{k-1} \mathbf{h}^k = 
  -\mathbf{r}^{k-1} + \mathbf{J}^{k-1} \mathbf{h}^{k-1}
\end{equation}

\begin{equation}
  r_n = \sum_{m \in \eta_n} Q_{n,m} + 
        Q_{n, \Gamma} - 
        Q_{\textrm{STO}}
\end{equation}

\begin{equation}
  J_{n,k} = \frac{\partial r_n}{\partial h_k}
\end{equation}

\subsection{Spatial Discretization and Internal Flows}
\subsection{Intercell Averaging of Conductance Terms}\label{sec-cond-avg}
\subsubsection*{Saturated Hydraulic Conductivity}
\subsubsection*{Relative Hydraulic Conductivity}
\subsection{Temporal Discretization and Storage}
\subsection{Sources and Sinks}
\subsection{Initial and Boundary Conditions}

\section{Integration into MODFLOW 6}
The capability presented in this report offers an alternative flow
formulation in the NPF package for cells that are under variably 
saturated conditions. The storage package (STO) has been extended
to include a more sophisticated treatment of partially saturated
cells. New packages have been developed to account for infiltration (INF),
ponding (PON), ... The existing packages for ... have been modified .

\section{Model Verification}
\subsection{Analytical Results}
\subsection{Example Problems}

\end{document}